\documentclass[aps,prc,reprint,amsmath]{revtex4-1}

\usepackage{hyperref}
\usepackage{lipsum}
\usepackage{graphicx}
\usepackage{pullquote}
\usepackage{floatflt}
\usepackage{tikz}
\usetikzlibrary{calc}
\graphicspath{{fig/}}

\newcommand{\trento}{T\raisebox{-0.5ex}{R}ENTo}
\newcommand{\avg}[1]{\langle #1 \rangle}
\newcommand{\nch}{N_\text{ch}}

\begin{document}

\title{Determining quark-gluon plasma initial conditions and medium \\ properties from model-to-data simulations with minimal bias}

\author{J.\ Scott Moreland}
\author{Jonah E.\ Bernhard}
\author{Steffen A.\ Bass}
\affiliation{Department of Physics, Duke University, Durham, NC 27708-0305}

\date{\today}


\begin{abstract}
  We implement a recently developed parametric initial condition model for relativistic nuclear collisions designed to interpolate among a general class of saturation based initial state models [?]. The model is compared to explicit calculations in Color-Glass Condensate effective field theory and embedded in a realistic hybrid model which couples event-by-event viscous hydrodynamics to a hadronic cascade in order to constrain free parameters of the model. We find that initial entropy deposition and particle production scale with the geometric mean of participant nuclear thickness $dS/dy \sim \sqrt{T_A T_B}$. 
\end{abstract}


\maketitle


\section{Introduction}

\lipsum[1-7]

\begin{figure*}
    \includegraphics{cgc_compare}
\end{figure*}

\lipsum[3-8]
\section{Section 2}


\begin{figure}[b]
    \includegraphics{thickness}
\end{figure}

\begin{figure}
    \includegraphics{nch_per_npart}
\end{figure}
\lipsum[5-10]

\begin{figure*}
    \includegraphics{observables_plot}
\end{figure*}
\lipsum[5-8]

\begin{figure*}
    \includegraphics{mode_observables}
\end{figure*}

\section{Section 2}
\lipsum[1-20]


\lipsum[3]


\lipsum[1-3]

%\begin{figure*}
\begin{pullquote}
{%
    shape=square, objdist=2mm, objvoffset=3, objvalign=bottom,%
    object=%
    {%
        \begin{tikzpicture}
            %\clip (-8,-8)--(-8,8)--(8,-8)--cycle;
            \node (0, 0) {\includegraphics[width=\textwidth]{posterior}};
            \clip (current bounding box.south west)--
                  (current bounding box.north west)--
                  ++(3,0)--
                  ($(current bounding box.south east)+(0,3)$)--
                  (current bounding box.south east)--
                  (current bounding box.south west);
        \end{tikzpicture}%
    }
}
\end{pullquote}
%\caption{blah blah blah blah asekgf liwulfasgfgla iliwfilwf wf}
%\end{figure*}

\lipsum[1-20]
\section{Section 3}

\begin{figure}
    \centering
    \includegraphics{posterior_p_arrows}
\end{figure}

\lipsum[3]

\begin{figure}
    \includegraphics{viscosity_samples}
\end{figure}

\lipsum[3]

\bibliography{trento2}


\end{document}
