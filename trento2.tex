\documentclass[aps,prc,reprint,amsmath]{revtex4-1}

\usepackage{hyperref}
\usepackage{graphicx}
\graphicspath{{fig/}}

\newcommand{\trento}{T\raisebox{-0.5ex}{R}ENTo}
\newcommand{\avg}[1]{\langle #1 \rangle}
\newcommand{\nch}{N_\text{ch}}

\begin{document}

\title{Extracting quark-gluon plasma initial conditions and medium properties \\ from model to data simulations with minimal bias}

\author{J.\ Scott Moreland}
\author{Jonah E.\ Bernhard}
\author{Steffen A.\ Bass}
\affiliation{Department of Physics, Duke University, Durham, NC 27708-0305}

\date{\today}


\begin{abstract}
  We implement a recently developed parametric initial condition model for relativistic nuclear collisions designed to interpolate among a general class of saturation based particle production scenarios [?]. 
  The model is compared to explicit calculations in Color-Glass Condensate effective field theory and embedded in a realistic event-by-event viscous hydrodynamics model with a hadronic afterburner to constrain free parameters of the model and  
\end{abstract}


\maketitle


%\section{Introduction}


%\cite{Moreland:2014oya}

\begin{figure*}
    \includegraphics{cgc_compare}
\end{figure*}

\begin{figure}
    \includegraphics{thickness}
\end{figure}

\bibliography{trento2}


\end{document}
