\documentclass[aps,prc,reprint,amsmath]{revtex4-1}

\usepackage{hyperref}
\usepackage{graphicx}
\usepackage{lipsum}
\graphicspath{{fig/}}

\newcommand{\trento}{T\raisebox{-0.5ex}{R}ENTo}
\newcommand{\avg}[1]{\langle #1 \rangle}
\newcommand{\nch}{N_\text{ch}}

\begin{document}

\title{Determining quark-gluon plasma initial conditions and medium \\ properties from model-to-data simulations with minimal bias}

\author{J.\ Scott Moreland}
\author{Jonah E.\ Bernhard}
\author{Steffen A.\ Bass}
\affiliation{Department of Physics, Duke University, Durham, NC 27708-0305}

\date{\today}


\begin{abstract}
  We implement a recently developed parametric initial condition model for relativistic nuclear collisions designed to interpolate among a general class of saturation based initial state models [?]. The model is compared to explicit calculations in Color-Glass Condensate effective field theory and embedded in a realistic hybrid model which couples event-by-event viscous hydrodynamics to a hadronic cascade in order to constrain free parameters of the model. We find that initial entropy deposition and particle production scale with the geometric mean of participant nuclear thickness $dS/dy \sim \sqrt{T_A T_B}$. 
\end{abstract}


\maketitle


\section{Introduction}

% introduce QGP and relativistic hydro
Measurements of the bulk properties of final state particles produced by ultra-relativistic nuclear collisions indicate that hot and dense QCD matter behaves as a 
strongly interacting system of deconfined quarks and gluons known as a quark-gluon plasma (QGP). Relativistic fluid dynamic simulations have been highly successful
in describing the large degree of collectivity observed in these collisions, most notably the existence of azimuthal correlations in particle yields understood to arise
when particles are emitted from a commonly flowing source.

% state the goal/value of hydro modeling, why do it?
Hydrodynamic models and related transport theories provide a necessary link between the ephemeral, early-time dynamics of the QGP droplet and the asymptotic 
bound states measured by experiment. Provided realistic initial conditions, the QCD equation of state and values for the dissipative transport coefficients, the hydrodynamic equations of motion specify the full time-evolution the QGP medium which can be used to evolve the.

\begin{figure*}
    \includegraphics{cgc_compare}
    \caption{Profiles of the initial thermal distribution predicted by the KLN (left), EKRT (middle) and wounded nucleon (right) models compared to a generalized mean with different values of the parameter $p$. Lines show different cross sections of the initial entropy density $dS/(d^2r_\perp dy)$ as a function of the nucleon density $T_A$ for several values of $T_B = 1, 2, 3$ [fm$^{-2}$].} 
\end{figure*}

\begin{figure}
    \includegraphics{trento_events}
    \caption{Several \protect\trento\ Pb+Pb initial condition events for the transverse entropy density $dS/d^2r_\perp dy$ calculated using generalized mean parameter $p=0$, nucleon width $w=0.5$ fm and gamma fluctuation factor $k=1.4$.}
\end{figure}

\begin{figure}[b]
    \includegraphics{thickness}
\end{figure}

\begin{figure}
    \includegraphics{nch_per_npart}
\end{figure}

\begin{figure*}
    \includegraphics{observables_plot}
\end{figure*}

\begin{figure*}
    \includegraphics{validation}
\end{figure*}

\begin{figure*}
    \includegraphics{mode_observables}
\end{figure*}

\begin{figure*}
    \includegraphics{posterior_identified}
    \caption{Bayesian posterior}
\end{figure*}

\begin{figure*}
    \includegraphics{posterior_compare}
\end{figure*}

\begin{figure}
    \centering
    \includegraphics{posterior_p_arrows}
\end{figure}

\begin{figure}
    \includegraphics{viscosity_samples}
\end{figure}

\appendix
\begin{figure*}
    \includegraphics{posterior_integrated}
    \caption{Bayesian posterior}
\end{figure*}

\bibliography{trento2}


\end{document}
