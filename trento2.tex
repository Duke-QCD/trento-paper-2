\documentclass[aps,prc,reprint,amsmath,nofootinbib]{revtex4-1}

\usepackage{hyperref}
\usepackage{graphicx}
\usepackage{lipsum}
\usepackage{amsmath}
\usepackage{amssymb}
\usepackage[inline]{enumitem}
\graphicspath{{fig/}}

\newcommand{\trento}{T\raisebox{-0.5ex}{R}ENTo}
\newcommand{\avg}[1]{\langle #1 \rangle}
\newcommand{\nch}{N_\text{ch}}
\newcommand{\npart}{N_\text{part}}
\newcommand{\sqrts}{\sqrt{s_{NN}}}
\newcommand{\TA}{T_{\text{part},A}}
\newcommand{\TB}{T_{\text{part},B}}
\newcommand{\Qmin}{Q_{s,\text{min}}}
\newcommand{\Qmax}{Q_{s,\text{max}}}

\begin{document}

\title{Determining quark-gluon plasma initial conditions and medium \\ properties from model-to-data simulations with minimal bias}
\author{J.\ Scott Moreland}
\author{Jonah E.\ Bernhard}
\author{Steffen A.\ Bass}
\affiliation{Department of Physics, Duke University, Durham, NC 27708-0305}
\date{\today}


\begin{abstract}
We aim to determine the quark-gluon-plasma initial conditions created in ultra-relativistic heavy-ion collisions and its temperature dependent transport coefficients utilizing Bayesian statistics and a multi-parameter model-to-data comparison.
The study is performed using \trento\ a recently developed parametric initial condition model for relativistic nuclear collisions which is able to interpolate among a general class of saturation based particle production scenarios [?].
The parameterization is compared to explicit calculations in Color-Glass Condensate (CGC) effective field theory and embedded in a realistic hybrid model which couples event-by-event viscous hydrodynamics to a hadronic cascade.
We find that the parameterized initial conditions are highly constrained by bulk observables and provide a first set of constraints on the temperature dependence of the specific shear- and bulk-viscosities of the quark-gluon plasma.
\end{abstract}


\maketitle

% Introduction
\section{Introduction}

Simulations based on relativistic viscous hydrodynamics have been highly successful describing a wealth of bulk observables in heavy-ion collisions at the Relativistic Heavy-Ion Collider (RHIC) in Brookhaven, NY and the Large Hadron Collider (LHC) in Geneva, Switzerland.
Initially, the success of hydrodynamic simulations was primarily qualitative. The framework elegantly described a number experimental phenomena, for example the existence of large azimuthal particle correlations, the mass ordering of these correlations and their characteristic momentum dependence.

Modern hydrodynamic simulations have greatly improved upon the successes of first-generation models.
The addition of dissipative corrections to ideal hydrodynamics, event-by-event fluctuations in the colliding nuclei and modern lattice quantum chromodynamics (QCD) calculations for the quark-gluon plasma (QGP) equation of state are just a few examples of developments which have dramatically improved the agreement of hydrodynamic models with experiment.

These developments have positioned hydrodynamic modeling to evolve beyond a qualitative science and extract intrinsic properties of hot and dense QCD matter with quantitative uncertainties. A primary goal of the ongoing effort is to determine the temperature dependence of the QGP transport coefficients such as the specific shear viscosity $\eta/s$ which is theorized to reach a lower bound ${(\eta/s)(T) \ge 1/4\pi}$ near the QGP phase transition temperature. 

A prominent source of uncertainty in determining QGP transport coefficients from hydrodynamic model-to-data comparison is in modeling the initial stages of the collision.
Model predictions vary with the choice of initial conditions and hence prefer hydrodynamic transport coefficients, e.g.\ QGP shear and bulk viscosities, which differ from calculation to calculation.
For example, an estimate of the effective (constant) QGP shear viscosity $(\eta/s)_\text{QGP}$ needed to fit spectra and flows at RHIC found ${1/4\pi \le (\eta/s)_\text{QGP} \le 2.5/4\pi}$, where the uncertainty was largely attributable to a factor of two discrepancy in $(\eta/s)_\text{QGP}$ extracted with two different hydrodynamic initial condition models. 

The ongoing effort to constrain current bounds on the QGP shear viscosity typically focuses on improving theoretical descriptions of the initial conditions as well as the addition of new sensitive observables to assess the validity of each model's inherent assumptions and approximations.
The process thus defines an iterative cycle in which theory calculations are embedded in hydrodynamic transport simulations and analyzed against a comprehensive list of bulk observables to determine the likelihood distributions of the model parameters.

Model optimization and comparison is complicated by the high-dimensional nature of the underconstrained parameter space.
In addition to free parameters which describe the QGP transport coefficients, hydrodynamic simulations depend on a number of auxiliary parameters such as an effective nucleon width, QGP thermalization time and chemical freezeout temperature which require simultaneous optimization.
Evaluating hydrodynamic models at a single point in parameter space may require tens of thousand of hydrodynamic simulations and thus brute force optimization techniques quickly become intractable. 

One solution to the model optimization problem is the use of modern Bayesian methods designed to tune computationally intensive models which depend on free parameters that live in large and highly correlated parameter spaces.
A given model is first evaluated using a relatively small number of parameter configurations corresponding to discrete points in parameter space, and the model evaluations are used to train a Gaussian process emulator which interpolates model predictions between the training points.
A standard Markov chain Monte Carlo algorithm is then be used to explore the parameter space interpolated by the emulator and isolate regions of parameter space which optimally replicate experiment. 

Bayesian methods have been applied in previous studies to hydrodynamic simulations initialized with a two-component Monte Carlo Glauber (MC-Glb.) model and initial conditions derived from Color Glass Condensate effective field theory using the Monte Carlo Kharzeev-Levin-Nardi (MC-KLN) model.
Future studies might expand this coverage to more modern calculations of the QGP initial conditions in order to systematically constrain hydrodynamic transport coefficients and auxiliary parameters for each initialization model described in the literature.
Once the models are appropriately optimized, the accuracy of the different theory calculations can be assessed using an appropriately chosen goodness of fit measure or, as typically used in Bayesian methodology, an integrated likelihood that a given model describes experiment. 

An alternative approach to this method of direct model validation is to parameterize the QGP initial conditions and apply Bayesian parameter estimation to determine systematic constraints on the initial energy (or entropy) deposited by competing models.
The approach was first introduced in a comprehensive Bayesian analysis of heavy-ion collisions at RHIC and the LHC which constrained parameterized forms of the event-averaged initial conditions and temperature dependence of the QGP equation of state using viscous hydrodynamics coupled to a hadronic afterburner.
The authors parameterized the initial conditions using a saturation inspired mapping similar to an optical wounded nucleon model but with additional parameters to incorporate modifications to particle production expected in Color-Glass Condensate effective field theory.

In this work we extend previous efforts to parameterize and constrain the QGP initial conditions using \trento , a recently developed parametric initial condition model designed to interpolate a subspace of all initialization models, including, but not limited to, parameterizations that mimic specific calculations in Color-Glass Condensate effective field theory.
The parametric model generalizes trivially to incorporate fluctuations of the nucleon coordinates and minimum bias proton-proton yields and is used to conduct the first Bayesian analysis of QGP initial conditions which parameterizes both the magnitude of event-by-event fluctuations and the strength of saturation effects which modulate average particle production.
We show that the experimental data imparts simultaneous constraints on the QGP initial conditions and medium properties, and compare these constraints to specific calculations in the literature.  

% Modeling the QGP evolution
\section{Modeling the QGP evolution}


The heavy-ion collision transport dynamics are modeled in a multi-stage approach which uses relativistic viscous hydrodynamics for the time-evolution of the QGP medium and a microscopic Boltzmann equation to simulate the dynamics of the system after hadronization. 

% Hydrodynamics and Boltzmann transport
\subsection{Hydrodynamics and Boltzmann transport}

The hydrodynamic evolution is solved using an improved version of VISH2+1, a stable, extensively tested implementation of boost-invariant viscous hydrodynamics which has been updated to handle fluctuating event-by-event initial conditions and incorporate bulk viscous corrections with shear and bulk coupling.
The equations of motion are described by the second-order Israel-Stewart equations in the 14-moment approximation which yields a set of relaxation equations
\begin{subequations}
    \label{relaxation}
    \begin{align}
        \tau_\Pi \Pi + \dot{\Pi} =& -\zeta \theta - \delta_{\Pi\Pi} \Pi\theta + \phi_1 \Pi^2 + \lambda_{\Pi\pi} \pi^{\mu\nu} \sigma_{\mu\nu} \nonumber \\ 
        &+\phi_3 \pi^{\mu\nu}\pi_{\mu\nu}, \\
        \tau_\pi \dot{\pi}^{\langle \mu\nu \rangle} + \pi^{\mu\nu} =& ~2\eta\sigma^{\mu\nu} + 2\pi_\alpha^{\langle \mu} w^{\nu \rangle \alpha} - \delta_{\pi\pi} \pi^{\mu\nu} \theta \phi_7    \pi_\alpha^{\langle \mu} \pi^{\nu \rangle \alpha} \nonumber \\
&-\tau_{\pi\pi} \pi_\alpha^{\langle \mu}\sigma^{\nu \rangle \alpha} + \lambda_{\pi\Pi} \Pi \sigma^{\mu\nu} + \phi_6 \Pi \pi^{\mu\nu},
    \end{align}
\end{subequations}
for the bulk viscous pressure $\Pi$ and shear-stress tensor $\pi^{\mu\nu}$.
The terms $\zeta$ and $\eta$ are the first-order bulk and shear transport coefficients, and the parameters $\tau_\Pi$ and $\tau_\pi$ their second-order bulk and shear relaxation times.
The remaining transport coefficients multiplying shear and bulk terms on the right can then be expressed as functions of the relaxation times and other thermodynamic quantities and have been determined in the limit of small but finite masses in reference \ref{?}.

The relaxation equations in Eq.\eqref{relaxation} are then used to solve the hydrodynamic evolution equations,
\begin{equation}
    \label{conservation}
    \partial_\mu T^{\mu\nu} = 0, \quad T^{\mu\nu} = e u^\mu u^\nu  - \Delta^{\mu\nu} (P_0 + \Pi) + \pi^{\mu\nu},
\end{equation}
for the energy-momentum tensor $T^{\mu\nu}$, provided a set of initial conditions for the energy density $e(x,y)$ and four-velocity $u^\mu(x,y)$ as well as an equation of state to interrelate the energy density $e$, ideal pressure $P_0$ and temperature $T$ of each fluid cell in local thermal equilibrium.

We use a modern QCD equation of state (EoS) based on continuum extrapolated lattice calculations at zero baryon density published by the HotQCD collaboration and blended into a hadron resonance gas EoS in the interval $110 \le T \le 130$~MeV using a smoothstep interpolation function.
The HotQCD EoS, characterized by the parameterized interaction measure $\theta^{\mu\mu}(T) = (e - 3p)/T^4$, has been compared to additional state-of-the-art calculations by the Wuppertal-Budapest collaboration and shown to agree within published errors.
The two parameterizations were also studied in a recent error analysis at RHIC energies which quantified the effect of systematic lattice EoS discrepancies and statistical continuum extrapolation errors on hydrodynamic observables.
The effect of these errors on mean $p_T$, elliptic flow $v_2$ and triangular flow $v_3$ was found to be order 1\% and hence is expected to be negligible in the present analysis.

We seek to constrain, among other quantities, the temperature dependence of the shear and bulk transport coefficients $\eta(T)$ and $\zeta(T)$ in Eq.~\eqref{relaxation}.
The QGP specific shear viscosity $\eta/s$ is qualitatively expected to reach a global minimum near the QGP phase transition temperature and hence is parameterized using a simple form,
\begin{equation}
    (\eta/s)(T) = 
    \begin{cases}
        (\eta/s)_\text{min} + (\eta/s)_\text{slope} \cdot (T - T_c) &T > T_c \\
        (\eta/s)_\text{hrg} &T \le T_c
    \end{cases}
\end{equation}
where we fix $T_c = 0.155$~GeV using the approximate pseudo-critical phase transition temperature of the HotQCD equation of state, and $(\eta/s)_\text{min}$, $(\eta/s)_\text{slope}$ and $(\eta/s)_\text{hrg}$ are constants which we wish to determine.
For the temperature dependent bulk viscosity, we use the parameterization constructed in Ref.~\ref{?},
\begin{equation}
    \label{zetas}
    (\zeta/s)(T) = 
    \begin{cases}
        \begin{aligned}
            C_1 &+ \lambda_1 \exp [(x-1)/\sigma_1]  \\ &+ \lambda_2 \exp [ (x-1)/\sigma_2]
        \end{aligned}
        &T < T_a \\[3ex]
        A_0 + A_1 x + A_2 x^2 &T_a \le T \le T_b \\[2ex]
        \begin{aligned}
            C_2 &+ \lambda_3 \exp [-(x-1)/\sigma_3]  \\ &+ \lambda_4 \exp [-(x-1)/\sigma_4]
        \end{aligned}
        &T > T_b
    \end{cases}
\end{equation}
with dimensionless parameter $x = T/T_0$ and parameterization coefficients
\begin{align*}
    &C_1=0.03, ~~C_2=0.001, \\ 
    &A_0=-13.45, ~~A_1=27.55, ~~A_2=-13.77, \\
    &\sigma_1=0.0025, ~~\sigma_2=0.022, ~~\sigma_3=0.025, ~~\sigma_4=0.13, \\
    &\lambda_1=0.9, ~~\lambda_2=0.22, ~~\lambda_3=0.9, ~~\lambda_4=0.25, \\
    &T_0 = 0.18 \text{ GeV}, ~~T_a = 0.995\, T_0, ~~T_b = 1.05\, T_0.
\end{align*}
The $(\zeta/s)(T)$ curve in Eq.~\eqref{zetas} is then rescaled by an overall normalization factor $(\zeta/s)_\text{norm}$ which is a free parameter to be constrained by data. 

The hybrid model used in the present analysis switches from hydrodynamic field equations to a microscopic kinetic description of the medium once the fluid expands, cools and hadronizes. Kinetic descriptions are well suited to describe late stages of the collision where the inter-particle mean free path approaches the system size and approximate local thermal equilibirum breaks down.
Kinetic models also naturally describe species dependent kinetic freezeout and hadronic feed down dynamics which are difficult to incorporate in purely hydrodynamic models.
More detailed discussion of the advantages of hybrid models can be found in references \ref{?}.

The transition from viscous hydrodynamics to a hadronic cascade is performed along a constant isotherm $T_\text{sw}$ using the iSS particle sampler according to the Cooper-Frye prescription,
\begin{equation}
    \label{cooper_frye}
    \frac{dN}{dy p_T dp_T d\phi_p} = \int_\Sigma \frac{g}{(2\pi)^3} (f_0(x,p) + \delta f(x,p)) p^\mu d^3 \sigma_\mu,
\end{equation}
where $E$ and $p$ are the energy and momentum of the sampled particle, $f_0$ the ideal distribution function, $\delta f$ its viscous corrections and where $d^3\sigma_\mu$ is an element of the isothermal freezeout surface $\Sigma$.

In general, the viscous corrections to the distribution function $\delta f(x, p)$ in Eq.~\eqref{cooper_frye} include both shear and bulk contributions.
The shear correction to the distribution function is given by,
\begin{equation}
    \delta f_\text{shear}(x^\mu, p^\mu) = f_0(x^\mu, p^\mu)(1 \pm f_0(x^\mu, p^\mu))\frac{\pi_{\mu\nu}p^\mu p^\nu}{2(e+P)T^2}
\end{equation}
where $f_0$ is the ideal distribution function, $\pi_{\mu\nu}$ the shear stress tensor, $p^\mu$ the particle momentum and $e$, $P$ and $T$ the fluid's energy density, pressure and temperature.
The bulk viscous correction, on the other hand, is described by multiple formulations in the literature which predict significantly different behavior when either the bulk viscosity $\zeta$ or the transverse momentum $p_T$ are large.
We note, however, that since the bulk parameterization $(\zeta/s)(T)$ in Eq.~\eqref{zetas} is small in the allowed hydro-to-micro transition region $135 \le T_\text{sw} \le 165$~MeV used in this study, the bulk viscous corrections to the distribution function $\delta f_\text{bulk} \propto (\zeta/s)\vert_{T_\text{sw}}$ are also expected to be small and are neglected in the present work.

Once the fluid is converted into hadrons, the subsequent microscopic dynamics are modeled using the UrQMD hadronic afterburner.
The UrQMD model propagates all produced hadrons along classical trajectories, samples their stochastic binary collisions and simulates their resonance decays until the system becomes too dilute to continue interacting.
The calculation is performed using a Monte Carlo realization of the Boltzmann equation
\begin{equation}
    \frac{df_i(x,p)}{dt} = \mathcal{C}_i(x, p)
\end{equation}
where $f_i(x,p)$ represents the phase space density of species $i$ and $\mathcal{C}_i$ is its corresponding collision kernel.
Once the particles cease interacting, the position, momentum and particle type is saved to file.

% Initial state models vs. parametric initial conditions
\subsection{Parameterizing the initial conditions}


The hydrodynamic equations of motion necessitate initial conditions for the energy density $e$, four-velocity $u^\mu$, bulk pressure $\Pi$ and shear stress tensor $\pi^{\mu\nu}$ at the thermalization time $\tau_0$ when the fluid attains local thermal equillibrium. The initial conditions are typically procured using one of two types of models:
\begin{enumerate*}[label={\arabic*)}]
    \item dynamical models derived from approximations of QCD or related quantum field theories which describe the full time evolution of the system as it approaches local thermal equilibrium, and 
    \item simpler non-dynamical models which map the state of the system at time $\tau=0^+$ directly to time $\tau=\tau_0$ and set the initial flow profile $u^\mu$ to zero.
\end{enumerate*}

The importance of pre-equilibrium dynamics was quantified in a recent study by initializing hydrodynamic simulations with a free streaming phase (zero coupling) and switching to hydrodynamics (strong coupling) after different periods of time. The authors showed that although free streaming never leads to thermalization, it can be used to bracket the influence of pre-equilibrium dynamics on the medium evolution as the pre-equilibrium coupling strength is expected to fall between the free streaming and hydrodynamic limits. The analysis found a preference for a brief free streaming phase ${\tau_\text{fs} \lesssim 1}$~fm/c, but the effect on hydrodynamic bulk observables was small and modifications to the preferred value of the QGP specific shear viscosity $\eta/s$ were less than 10\%. In reality, the magnitude of the effect is expected to be even smaller as the pre-equilibrium coupling strength is necessarily non-zero.

We choose to neglect pre-equilibrium dynamics in the present study, and set the fluid four-velocity to zero ${u^\mu = (1,0,0,0)}$ as well as the shear and bulk viscous tensors $\pi^{\mu\nu}$ and $\Pi$ which quickly relax to their Navier-Stokes values. This reduces the initial conditions to a thermal energy density $e(\tau_0, x, y, \eta)$ which can be expressed as an entropy density via the QCD EoS. The initial conditions in boost-invariant hydrodynamics can thus described by an entropy density profile
\begin{equation}
    \label{mapping}
    s(\tau_0, x, y)\vert_{\eta=0} = f(\TA, \TB)
\end{equation}
where the mapping $f$ is some static function of the density of nuclear matter in each nucleus which participates in inelastic collisions denoted $T_{\text{part}A(B)}$, constructed by summing over the participant proton thicknesses
\begin{equation}
    \label{participant}
    T_{\text{part}}(\vec{x}) = \sum\limits_{i=1}^{N_\text{part}} \gamma_i\, T_p(\vec{x} - \vec{x}_i),
\end{equation}
using a Gaussian for the proton thickness function
\begin{equation}
    T_p(\vec{x}) = \frac{1}{2\pi w^2} \exp\bigg(\!-\frac{x^2 + y^2}{2 w^2}\bigg). 
\end{equation}    
with nucleon width $w$ taken as a free parameter. The random weight factor $\gamma_i$ multiplying each nucleon in Eq.~\eqref{participant} is added to account for minimum bias proton-proton multiplicity fluctuations and is sampled from a Gamma distribution $X = \Gamma(\mu, k)$ with unit mean $\mu$ and shape parameter $k = 1/\text{Var}(X)$.

The inelastic nucleon-nucleon participation probability, and hence the density of participant matter, can be calculated from a standard Glauber formalism,
\begin{equation}
    P_\text{coll}(b) = 1 - \exp \bigg[ -\sigma_{gg} \int d^2x_\perp T_p(|\vec{x}|) T_p(|\vec{x} - \vec{b}|) \bigg ] 
\end{equation}
where $b$ is the impact parameter between two nucleons and the partonic cross section $\sigma_{gg}$ is tuned to fit the inelastic nucleon-nucleon cross section
\begin{equation}
    \int 2 \pi b\, db\, P_\text{coll}(b) = \sigma_\text{NN}^\text{inel} 
\end{equation}
at the desired beam energy. The collision cross sections thus enter through the calculation of $\TA$ and $\TB$ which are determined \emph{independently} of the function $f$ used to convert local participant density into entropy deposition.

At this point it becomes necessary to specify the function $f$ in Eq~\eqref{mapping} which maps participant matter to local entropy deposition. In general, the function $f(\TA, \TB)$ is assumed to be symmetric, monotonically increasing and smooth. In this work we employ a parameterization for $f$ motivated in Ref.~\ref{?} based on a functional form known as the generalized mean,
\begin{align}
    \label{entropy_propto}
    s(\tau_0, x, y)\vert_{\eta=0} &\propto f(p; \TA, \TB), \\
    f(p; x, y) &\equiv \left( \frac{x^p + y^p}{2} \right)^{1/p}.
\end{align}
The parameterization introduces two free parameters, a normalization prefactor and a continuous entropy deposition parameter $p\in(-\infty, \infty)$ which interpolates between different types of entropy deposition schemes. For ${p=(-1, 0, 1)}$, the generalized mean reduces to the well known harmonic $2 x y/(x + y)$, geometric $\sqrt{x\,y}$ and arithmetic $(x + y)/2$ forms, and for $p\rightarrow \pm\infty$ it asymptotes to $\min(x,y)$ and $\max(x,y)$ functions.

\begin{figure}[t]
    \includegraphics{thickness}
    \caption{Cross section of the participant nucleon density in a mid-central Pb+Pb collision at $\sqrts=2.76$ TeV as a function of the transverse coordinate $x$ parallel to impact parameter $\vec{b}$. The gray band indicates the region bounded by the minimum and maximum values of the local participant thickness functions $T_A$ and $T_B$, while the blue band indicates the region spanned by the generalized mean of $T_A$ and $T_B$ with parameter $-1<p<1$. The solid blue line shows an example of a discrete mapping specified by a generalized mean with $p=0$.} 
    \label{fig:thickness}
\end{figure}

In Ref.~\ref{?} we motivate the generalized mean ansatz using basic physical constraints and phenomenological observations, but perhaps the simplest explanation of the ansatz is to examine the effect of the mapping on a realistic event. Fig.~\ref{fig:thickness} shows a cut-away of a lead-lead collision at $\sqrts=2.76$~TeV, sliced along the direction of impact parameter $\hat{b}$. At each point in the transverse plane there are two relevant scales of interest, the minimum participant density $T_\text{min}$ and the maximum participant density $T_\text{max}$. The gray band marks the region spanned by $T_\text{min}$ and $T_\text{max}$, while the blue band/line show the generalized mean acting on the participant density of each nucleus with different values of the parameter $p$. The figure shows that decreasing $p$ pulls the generalized mean towards the minimum of $\TA$ and $\TB$ while increasing $p$ pushes it to the maximum of the two. The generalized mean ansatz thus parameterizes asymmetric entropy deposition, or in the parlance of Color-Glass Condensate theory, the intensity of saturation effects on local gluon production.

At this point the reader may complain that while the generalized mean parameterizes entropy deposition in asymmetric regions of the collision $T_\text{min} < T_\text{max}$, it asserts a particular scaling law for entropy deposition in symmetric regions of the collision where $T_\text{min} = T_\text{max}$. In particular, 
\begin{equation}
    \label{homogenous}
    f(p; \alpha T_\text{part}, \alpha T_\text{part}) = \alpha T_\text{part}. 
\end{equation}
This property of the generalized mean is known as scale invariance or homogeneity. While it is difficult to empirically prove or disprove exact scale invariance, multiple experimental observations indicate that it holds to very good approximation. For example, it was demonstrated that collisions of highly deformed uranium nuclei exhibit elliptic flow patterns which are incompatible with a scale-violating binary collision term postulated by the two-component Glauber ansatz. Measurements of central copper-copper, gold-gold and uranium-uranium particle production at RHIC also exhibit approximate participant scaling in agreement with Eq.~\eqref{homogenous}. Moreover, the scale invariant constraint serves as a reasonable approximation for a number of calculations of the mapping $f$ in Eq.~\eqref{mapping} derived from CGC effective field theory as we show momentarily. At present we thus assert the scale invariant constraint as a simplifying postulate although future work should relax this constraint by including additional initial condition model parameters.  

The aforementioned construction defines the \trento\ initial condition model proposed in Ref.~\ref{?} and can be used to generate a wide range initial condition models by varying the four parameters which specify the entropy normalization, deposition parameter $p$, proton-proton multiplicity fluctuation parameter $k$ and nucleon width $w$. A few sample events of the model output are pictured in Fig.~\ref{fig:trento_events}. 

\begin{figure}
    \includegraphics{trento_events}
    \caption{Several \protect\trento\ Pb+Pb initial condition events for the transverse entropy density $dS/(d^2r_\perp dy)$ calculated using generalized mean parameter $p=0$, nucleon width $w=0.5$~fm and gamma fluctuation factor $k=1.4$.}
    \label{fig:trento_events}
\end{figure}

\subsection{Mimicing initial condition models}

\begin{figure*}
    \includegraphics{cgc_compare}
    \caption{Profiles of the initial thermal distribution predicted by the KLN (left), EKRT (middle) and wounded nucleon (right) models (dashed black lines) compared to a generalized mean with different values of the parameter $p$ (solid blue lines).
    Staggered lines show different cross sections of the initial entropy density $dS/(d^2r_\perp dy)$ as a function of the nucleon density $T_A$ for several values of $T_B = 1, 2, 3$ [fm$^{-2}$].} 
\end{figure*}

Charged particle production in the KLN model can be approximated from the produced gluon density $\nch \propto N_g$ where,
\begin{equation}
    \label{kln}
    \frac{dN_g}{dy\,d^2r_\perp} \sim \Qmin^2 \biggl[ 2 + \log \biggr(\frac{\Qmax^2}{\Qmin^2} \biggr) \biggr]
\end{equation}
and $\Qmax$ and $\Qmin$ denote the larger and smaller values of the two saturation scales in opposite nuclei at any fixed position in the transverse plane.
In the original formulation of the KLN model, the two saturation scales are proportional to the local participant nucleon density in each nucleus $Q^2_{s,A} \propto \TA$, which can be used to recast Eq.~\eqref{kln} as a function of participant nucleon density,
\begin{align}
    s(\tau_0, x, y)\vert_{\eta=0} &\propto f(\TA, \TB), \\
    f(x, y) &= \min(x, y)\biggr[2 + \log \biggr(\frac{\max(x,y)}{\min(x,y)}\biggr) \biggr].
\end{align}

Particle production in the EKRT model can be approximated from the initial entropy density ${\nch \propto \int dx\,dy S(x,y)}$ where,
\begin{align}
    p_\text{sat}(\rho_{AA}) &= C(a + T_A T_B)^n - b\,C a^n \\
    e_0 &= \frac{K_\text{sat}}{\pi} p_\text{sat}^3 \\
    s_0 &= e_0^{3/4} 
\end{align}
with coefficients,
\begin{align*}
    &C_1=0.03, ~~C_2=0.001, \\ 
    &A_0=-13.45, ~~A_1=27.55, ~~A_2=-13.77, \\
    &\sigma_1=0.0025, ~~\sigma_2=0.022, ~~\sigma_3=0.025, ~~\sigma_4=0.13, \\
    &\lambda_1=0.9, ~~\lambda_2=0.22, ~~\lambda_3=0.9, ~~\lambda_4=0.25, \\
    &T_0 = 0.18 \text{ GeV}, ~~T_a = 0.995\, T_0, ~~T_b = 1.05\, T_0.
\end{align*}

\vfill


\section{Constraining the model}

\begin{figure*}
    \includegraphics{validation}
    \caption{Emulator validation for pion yield $dN/dy_{\pi^\pm}$ (left), average transverse momentum ${\langle p_T \rangle}_{\pi^\pm}$ (middle) and flow cumulant $v_2\{2\}$ (right) for centralities 0--5\% (blue) and 30--40\% (orange). The horizontal axis shows predictions by the emulator at untested regions of parameter space, and the vertical axis shows the same quantity after it has been calculated directly from the model. The gray line indicates perfect model and emulator agreement.} 
\end{figure*}

\begin{figure*}
    \includegraphics{posterior_identified}
    \caption{Diagonal and lower-diagonal: posterior marginal and joint distributions of the calibration parameters described in section \ref{?}. Diagonal entries show the posterior likelihood distribution of each parameter marginalized over remaining parameters, while lower-diagonal entries show the joint distribution of parameter pairs. Upper diagonal: the Pearson product-moment correlation coefficient $\rho_{X,Y}=\text{cov}(X,Y)/(\sigma_X \sigma_Y)$ for each pair of parameters $X$, $Y$. Red (blue) tiles indicate Pearson coefficients close to $1$ ($-1$) corresponding to strong linear correlation (anti-correlation). $^\dagger$The units for $\eta/s$ slope are [GeV$^{-1}$].}
    \label{fig:posterior_identified}
\end{figure*}

\begin{figure*}
    \includegraphics{posterior_compare}
    \caption{Comparison of the posterior distributions obtained when the model is calibrated to fit identified pion, kaon and proton yields $dN/dy$ (blue lines), and when the model is calibrated to fit integrated charged particles $d\nch/d\eta$ (orange lines). $^\dagger$The units for $\eta/s$ slope are [GeV$^{-1}$].}
\end{figure*}

\begin{figure*}
    \includegraphics{observables_plot}
    \caption{Identified yields (left column), mean $p_T$ (middle column) and flow cumulants $v_n\{2\}$ (right column). The top row shows results obtained from the training data used to condition the emulator. from 100 random samples drawn from the Bayesian posterior.}
\end{figure*}

% Results
\section{Results}

\begin{table}
    \caption{High likelihood parameters selected from the posterior distribution in Fig.~\ref{fig:posterior_identified} used to generate Fig.~\ref{fig:mode_observables}}
    \begin{ruledtabular}    
        \begin{tabular}{llll}
            \multicolumn{2}{c}{Initial condition} & \multicolumn{2}{c}{QGP medium} \\
            \noalign{\smallskip}\hline\noalign{\smallskip}
            norm & 120.          &  $\eta/s$ min   & 0.08       \\
            $p$  & 0.0           &  $\eta/s$ slope & 0.85 GeV$^{-1}$   \\
            $k$  & 1.5           &  $\zeta/s$ norm & 1.25       \\
            $w$  & 0.43 fm       &  $T_\text{sw}$  & 0.148 GeV  \\
        \end{tabular}
    \end{ruledtabular}
    \label{table:likely_param}
\end{table}

\begin{figure*}
    \includegraphics{mode_observables}
    \caption{Top row: model calculations of identified yields $dN/dy$, mean $p_T$ and flow cumulants $v_n\{2\}$ as a function of collision centrality using high-likelihood parameters (listed in table \ref{table:likely_param}) determined from the posterior in Fig.~\ref{fig:posterior_identified}. Symbols with error bars show data from ALICE and lines show results of the model. Bottom row: ratio of the model divided by experiment. Gray band indicates 10\% discrepancy between the model and experiment.}
    \label{fig:mode_observables}
\end{figure*}


\subsection{Initial condition properties}
\begin{figure}
    \includegraphics{nch_per_npart}
    \caption{Average charged particle density per participant pair $(d\nch/d\eta)/(\npart/2)$ as a function of participant number for Pb+Pb and p+Pb collisions at $\sqrts=5.02$ TeV.
    Symbols are data from ALICE and lines are model calculations from \protect\trento\ using parameters selected by the Bayesian analysis.}
\end{figure}

\begin{figure}
    \centering
    \includegraphics{posterior_p_arrows}
    \caption{Posterior on the generalized mean parameter $p$ which modulates the intensity of saturation effects in the initial conditions compared to the effective p-values needed to fit
    the KLN, EKRT and wounded nucleon (WN) models.}
    \label{posterior_p_arrows}
\end{figure}

\begin{figure}
    \centering
    \includegraphics{ipglasma}
    \caption{Comparison of trento parameteric initial conditions
    Charged particle multiplicity $d\nch/d\eta$ (top) and eccentricity harmonics $\varepsilon_n$ (bottom) for ${\sqrt{s_{NN}}=2.76}$ TeV Pb+Pb collisions as a function of impact parameter $b$. \protect\trento\ results using parameters drawn from the Bayesian posterior in Fig.~\ref{fig:posterior_identified} (solid lines) compared to results from the IP-Glasma model. Each \protect\trento\ initial condition event is freestreamed an additional 0.4 fm/c to match the pre-equilibrium evolution time of Classical Yang-Mills dynamics in the IP-Glasma model.}
\end{figure}

\subsection{QGP medium properties}
\begin{figure}
    \includegraphics{viscosity_samples}
    \caption{Posterior constraints on the temperature dependent shear viscosity $(\eta/s)(T)$ (top panel) and temperature dependent bulk viscosity $(\zeta/s)(T)$ (bottom panel). The gray shaded regions mark the allowable values of $\eta/s$ and $\zeta/s$ used in the design while blue lines represent sample curves drawn from the posterior distributions of $\eta/s$ min, $\eta/s$ slope and $\zeta/s$ norm in Fig.~\ref{fig:posterior_identified}. unscaled bulk viscosity curve with unit norm (black line) used by the authors in \ref{?} is shown for reference.}
\end{figure}

\section{Conclusion}

\acknowledgements
The authors thank Ulrich Heinz, Scott Pratt, Harri Niemi and Bj\"orn Schenke for helpful discussions and clarifications in the production of the manuscript.
\appendix

\begin{figure*}
    \includegraphics{posterior_integrated}
    \caption{Presentation is identical to Fig.~\ref{fig:posterior_identified}, but with charged particle yields replacing identified pion, kaon and proton yields in the calibration process. The resulting posterior distributions are generally less constrained, and several marginal distributions along the diagonal are affected noticeably (see Fig.~\ref{fig:posterior_comparison} for a direct comparison). For example, the initial condition normalization increases significantly (top-left panel) and the hydro-to-urqmd switching temperature is completely washed out (bottom-right panel). $^\dagger$The units for $\eta/s$ slope are [GeV$^{-1}$]. }
\end{figure*}


\bibliography{trento2}


\end{document}
